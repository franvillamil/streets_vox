\documentclass[12pt, titlepage]{article}
\usepackage[bottom = 5cm, top = 5cm, left = 3cm, right = 3cm]{geometry}
\usepackage[english]{babel}
\usepackage[utf8]{inputenc}
\usepackage[table]{xcolor}
\usepackage{graphicx, booktabs, tikz, csquotes, subfig, enumitem, dcolumn, pdfpages}
\usepackage[font=normalsize]{caption}%,labelfont=bf
\usepackage{subcaption}

\usepackage{endnotes}
\let\footnote=\endnote

\renewcommand*\rmdefault{ppl}

\usepackage{setspace}
\setstretch{1.5}

\usepackage[]{titlesec}
    \titleformat*{\section}{\large\bf}
    \titleformat*{\subsection}{\normalsize\it}

% Bibliography
% \usepackage[natbibapa]{apacite}
\usepackage[round]{natbib}
% \renewcommand{\bibliographytypesize}{\normalsize}
\setlength{\bibsep}{5pt}

\usepackage[colorlinks = TRUE, allcolors = blue]{hyperref}

\widowpenalty=10000
\clubpenalty=10000

\title{\Large Do TJ policies cause backlash?\\Evidence from street name changes in Spain}
\author{Francisco Villamil\thanks{Juan March Institute--Carlos III University of Madrid} \and Laia Balcells\thanks{Georgetown University}}
\date{\today}

%\usepackage[none]{hyphenat}

\begin{document}

\maketitle

% \begin{abstract}
% \setstretch{1.2}
% Memories of old conflicts shape domestic politics long after these conflicts end.
% The debates about the Confederacy in the United States or the Francoist regime in Spain suggest that these are sensitive topics that might increase political polarization, particularly when transitional justice policies are implemented to address grievances. One such policy recently debated in the US and Spain is the removal of public symbols linked to past authoritarian regimes. However, the empirical evidence on their impact is still limited. This article attempts to fill this gap by exploring the impact of removing Francoist street names in Spain. Using cross-sectional and difference-in-differences analyses, we show that removing Francoist street names has increased electoral support to the new far-right party, Vox. Results suggest that revisiting the past and trying to redress the victims' grievances can cause a backlash among those ideologically aligned with the perpetrator.
%
% \vspace{10pt}
% \noindent
% \textbf{Keywords:} transitional justice, voting, conflict memories, Spain
%
% \end{abstract}

\end{document}
